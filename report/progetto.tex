\documentclass[a4paper, 11pt]{article}
\usepackage[margin=3cm]{geometry}
\usepackage[]{fontenc}
\usepackage[utf8]{inputenc}
\usepackage[italian]{babel}
\usepackage{geometry}
\geometry{a4paper, top=2cm, bottom=3cm, left=1.5cm, right=1.5cm, heightrounded, bindingoffset=5mm}
\usepackage{amsmath}
\usepackage{amssymb}
\usepackage{gensymb}
\usepackage{graphicx}
\usepackage{psfrag,amsmath,amsfonts,verbatim}
\usepackage{xcolor}
\usepackage{color,soul}
\usepackage{fancyhdr}
\usepackage{indentfirst}
\usepackage{graphicx}
\usepackage{newlfont}
\usepackage{amssymb}
\usepackage{amsmath}
\usepackage{latexsym}
\usepackage{amsthm}
\usepackage{mathtools}
%\usepackage{subfigure}
\usepackage{subcaption}
\usepackage{psfrag}
\usepackage{footnote}
\usepackage{graphics}
\usepackage{color}
\usepackage{hyperref}
\usepackage{tikz}


\usetikzlibrary{snakes}
\usetikzlibrary{positioning}
\usetikzlibrary{shapes,arrows}

	
	\tikzstyle{block} = [draw, fill=white, rectangle, 
	minimum height=3em, minimum width=6em]
	\tikzstyle{sum} = [draw, fill=white, circle, node distance=1cm]
	\tikzstyle{input} = [coordinate]
	\tikzstyle{output} = [coordinate]
	\tikzstyle{pinstyle} = [pin edge={to-,thin,black}]

\newcommand{\courseacronym}{CAT}
\newcommand{\coursename}{Linea Guida Report\\Controlli Automatici - T}
\newcommand{\tipology}{B}
\newcommand{\trace}{1}
\newcommand{\projectname}{Controllo di uno scaldatore elettrico}
\newcommand{\group}{23}
\newcommand{\myalpha}{\dfrac{h_RA_R}{m_Rc_R}}
\newcommand{\mybeta}{\dfrac{1}{m_Rc_R}}
\newcommand{\mygamma}{\dfrac{\dot{m}_A}{m_A}}
\newcommand{\myphi}{\dfrac{h_RA_R}{m_Ac_A}}

%opening
\title{ \vspace{-1in}
		\huge \strut \coursename \strut 
		\\
		\Large  \strut Progetto Tipologia \tipology - Traccia \trace 
		\\
		\Large  \strut \projectname\strut
		\\
		\Large  \strut Gruppo \group\strut
		\vspace{-0.4cm}
}
\author{Nobili Giacomo, Raffoni Federico, Roca Marco}
\date{}

\begin{document}

\maketitle
\vspace{-0.5cm}

Il progetto riguarda il controllo di uno scaldatore elettrico, la cui dinamica viene descritta dalle seguenti equazioni differenziali 
%
\begin{subequations}\label{eq:system}
\begin{align}
	m_Rc_R\frac{dT_R(t)}{dt} &= h_RA_R(T_{out}(t)-T_R(t)) + \frac{P_E(t)}{(1 + \kappa T_R(t))} \\
	m_Ac_A\frac{dT_{out}(t)}{dt} &= \dot{m}_Ac_A(T_{in}-T_{out}(t)) + h_RA_R(T_R(t)-T_{out}(t)),
\end{align}
\end{subequations}
%
dove

\begin{itemize}
	\item $T_R(t)$ è la temperatura del riscaldatore [C°];
	\item $T_{out}(t)$ è la temperatura dell'aria in uscita dal riscaldatore [C°];
	\item $P_E(t)$ è la potenza elettrica fornita [W];
	\item $T_{in}$ è la temperatura dell'aria in ingresso (ambiente a temperatura costante) [C°];
	\item $m_R$ è la massa del riscaldatore [kg];
	\item $c_R$ è il calore specifico del riscaldatore [J/(kg C°)];
	\item $h_R$ è il coefficiente di convezione tra riscaldatore e aria [m²];
	\item $\kappa$ è il coefficiente di variazione della resistenza con la temperatura [1/C°];
	\item $m_A$ è la massa dell'aria [kg];
	\item $c_A$ è il calore specifico dell'aria [J/(kg C°)];
	\item $\dot{m}_A$ è la portata massica dell'aria [kg/s].
\end{itemize}


\section{Espressione del sistema in forma di stato e calcolo del sistema linearizzato intorno ad una coppia di equilibrio}

Innanzitutto, esprimiamo il sistema~\eqref{eq:system} nella seguente forma di stato
%
\begin{subequations}
\begin{align}\label{eq:state_form}
	\dot{x} &= f(x,u)
	\\
	y &= h(x,u).
\end{align}
\end{subequations}
%
Pertanto, andiamo individuare lo stato $x$, l'ingresso $u$ e l'uscita $y$ del sistema come segue 
%
\begin{align*}
	x := \begin{bmatrix}
		T_R
		\\
		T_{out}
	\end{bmatrix}, \quad u := P_E, \quad y := T_{out}
\end{align*}
%
Coerentemente con questa scelta, ricaviamo dal sistema~\eqref{eq:system} la seguente espressione per le funzioni $f$ ed $h$
%
\begin{align*}
	f(x,u) &=  \begin{bmatrix}
		f_1 (x,u)
		\\
		f_2 (x,u)
	\end{bmatrix}  
    := 
    \begin{bmatrix}
		\myalpha x_2 - \myalpha x_1 + \dfrac{u}{(m_Rc_R)(1+\kappa x_1)} \\ \\
		\mygamma T_{in} - (\mygamma + \myphi) x_2 + \myphi x_1
	\end{bmatrix} 
	\\
	h(x,u) &:= x_{2}
\end{align*}
%
Una volta calcolate $f$ ed $h$ esprimiamo~\eqref{eq:system} nella seguente forma di stato
%
\begin{subequations}\label{eq:our_system_state_form}
\begin{align}
	\begin{bmatrix}
		\dot{x}_1
		\\
		\dot{x}_2
	\end{bmatrix} &=
	\begin{bmatrix}
		\myalpha x_2 - \myalpha x_1 + \dfrac{u}{(m_Rc_R)(1+\kappa x_1)} \\ \\
		\mygamma T_{in} - (\mygamma + \myphi) x_2 + \myphi x_1
	\end{bmatrix} 
	\label{eq:state_form_1}
	\\ \\
	y &= x_2
\end{align}
\end{subequations}
%
Per trovare la coppia di equilibrio $(x_e, u_e)$ di~\eqref{eq:our_system_state_form}, andiamo a risolvere il seguente sistema di equazioni
%
\begin{align}
	\begin{cases}
		f_1(x_e, u_e) = 0\\
		f_2(x_e, u_e) = 0
	\end{cases}
	\implies
	\begin{cases}
		\myalpha ( x_{2e} - x_{1e}) + \dfrac{u_e}{(m_Rc_R)(1 + \kappa x_{1e})} = 0\\ \\
		\mygamma T_{in} - (\mygamma + \myphi) x_{2e} + \myphi x_{1e} = 0
	\end{cases}
\end{align}\\
In particolare sono noti gli equilibri per lo stato $(x_{1e}, x_{2e})$ e dobbiamo ricavare $u_e$ in relazione ad essi.
Possiamo ottenere il risultato dalla prima equazione isolando $u_e$. Procedendo con i calcoli otteniamo:\\
\begin{align*}
	u_e = h_RA_R(x_{1e}-x_{2e})(1+\kappa x_{1e})
\end{align*}
%
Infine, sostituendo i parametri, la coppia di equilibrio risulta:
%
\begin{align}
	x_e := \begin{bmatrix}
		200 \text{ °C}\\
		28.8136 \text{ °C}
	\end{bmatrix},  \quad u_e = 1.0785 \cdot 10^3 \text{ W}.\label{eq:equilibirum_pair}
\end{align}
%
Definiamo le variabili alle variazioni $\delta x$, $\delta u$ e $\delta y$ come 
%
\begin{align*}
	\delta x \approx x(t) -x_e, 
	\quad
	\delta u \approx u(t) -u_e, 
	\quad
	\delta y \approx y(t) -y_e.
\end{align*}
%
L'evoluzione del sistema espressa nelle variabili alle variazioni pu\`o essere approssimativamente descritta mediante il seguente sistema lineare
%
\begin{subequations}\label{eq:linearized_system}
\begin{align}
	\delta \dot{x} &= A\delta x + B\delta u
	\\
	\delta y &= C\delta x + D\delta u,
\end{align}
\end{subequations}
%
dove le matrici $A$, $B$, $C$ e $D$ vengono calcolate come

%
\renewcommand{\arraystretch}{2}
\begin{subequations}\label{eq:matrices}
\begin{align}
	A &= 
    \begin{bmatrix}
    \left.\dfrac{\partial f_1(x,u)}{\partial x_1}\right|_{x_e,u_e} &
    \left.\dfrac{\partial f_1(x,u)}{\partial x_2}\right|_{x_e,u_e} \\
    \left.\dfrac{\partial f_2(x,u)}{\partial x_1}\right|_{x_e,u_e} &
    \left.\dfrac{\partial f_2(x,u)}{\partial x_2}\right|_{x_e,u_e}
    \end{bmatrix}
    = 
    \begin{bmatrix}
		-(\myalpha + \mybeta \dfrac{u_e\kappa}{(1+\kappa x_{1e})^2}) & \myalpha \\
		\myphi & - \myphi - \mygamma
	\end{bmatrix}\\
    &=
    \begin{bmatrix}
    -0.044 & -0.0035 \\
    -0.0428 & -1.9640
    \end{bmatrix}
	\\ \\
	B &=
    \begin{bmatrix}
    \left.\dfrac{\partial f_1(x,u)}{\partial u}\right|_{x_e,u_e} \\
    \left.\dfrac{\partial f_2(x,u)}{\partial u}\right|_{x_e,u_e}
\end{bmatrix}
    =
    \begin{bmatrix}
		\dfrac{1}{(m_Rc_R)(1+\kappa x_{1e})}\\
		0
	\end{bmatrix}
    =
    \begin{bmatrix}
    0.5624\cdot 10^{-3} \\
    0
    \end{bmatrix},
	\\ \\
	C &=
    \begin{bmatrix}
    \left.\dfrac{\partial h(x,u)}{\partial x_1}\right|_{x_e,u_e} &
    \left.\dfrac{\partial h(x,u)}{\partial x_2}\right|_{x_e,u_e}
    \end{bmatrix}
=
\begin{bmatrix}
0 & 1
\end{bmatrix}
	\\ \\
	D &=
    \begin{bmatrix}
    \left.\dfrac{\partial h(x,u)}{\partial u}\right|_{x_e,u_e}
    \end{bmatrix}
    =
    \begin{bmatrix}
    0
    \end{bmatrix}
\end{align}
\end{subequations}
\renewcommand{\arraystretch}{1}
%
\section{Calcolo Funzione di Trasferimento}

In questa sezione, andiamo a calcolare la funzione di trasferimento $G(s)$ dall'ingresso $\delta u$ all'uscita $\delta y$ mediante la seguente formula 
%
%
\begin{align*}
G(s) = C \dfrac{adj(sI-A)}{det(sI-A)}B + D
\end{align*}

Procediamo per passi e calcoliamo prima il determinante di $sI-A$:\\
\begin{align*}
	det(sI-A) = (s + \myalpha + \dfrac{u_e\kappa}{(m_Rc_R)(1+\kappa x_{1e})^2})(s + \mygamma + \myphi) - \dfrac{(h_RA_R)^2}{m_Am_Rc_Ac_R}
\end{align*}\\

poi la matrice aggiunta di $sI - A$. 
Visto che quest'ultima è una matrice $2x2$, vale la seguente:\\

$$ adj\Big(\begin{bmatrix}
	a & b\\
	c & d
\end{bmatrix}\Big) = \begin{bmatrix}
	d & -b\\
	-c & a
\end{bmatrix}$$

e quindi:

$$adj(sI-A) = \begin{bmatrix}
	s + \mygamma + \myphi & \myalpha\\ \\
	\myphi & s + \myalpha \dfrac{u_e\kappa}{(m_Rc_R)(1+\kappa x_{1e})^2}
\end{bmatrix}$$\\

Mettendo tutto insieme abbiamo:\\ 
\begin{align*}
	G(s) = \dfrac{1}{det(sI-A)}
	\begin{bmatrix} 0 & 1 \end{bmatrix}
	\begin{bmatrix}
		s + \mygamma + \myphi & \myalpha\\ \\
	\myphi & s + \myalpha \dfrac{u_e\kappa}{(m_Rc_R)(1+\kappa x_{1e})^2}
	\end{bmatrix}
	\begin{bmatrix}
		\dfrac{1}{(m_Rc_R)(1+\kappa x_{1e})}\\
		0
	\end{bmatrix}
\end{align*}\\
Infine, sostituendo i parametri e svolgendo i calcoli si ottiene la funzione di trasferimento del sistema:
\begin{align}\label{eq:transfer_function}
	\setlength{\fboxsep}{20pt}
	\Aboxed{ G(s) = \dfrac{2.407 \cdot 10^-5}{s^2 + 1.968s + 8.509 \cdot 10^-3} }
\end{align}\\
%
Dunque il sistema linearizzato~\eqref{eq:linearized_system}
è caratterizzato dalla funzione di trasferimento~\eqref{eq:transfer_function}
con \textbf{2 poli} e \textbf{nessuno zero}. I poli (indicati con $p_1$ e $p_2$) sono i seguenti: 
\begin{align*}
	p_1 &= -1.9641 \\ 
	p_2 &= -0.0043
\end{align*}\\
In Figura \ref{fig:bode_diagram} mostriamo il corrispondente diagramma di Bode.\\

\begin{figure}[h!]
    \centering
    \includegraphics [width=0.9\textwidth]{../figs/G_diagram.pdf} 
    \caption{Diagramma di Bode del sistema.}
    \label{fig:bode_diagram}
\end{figure}
\section*{Analisi in Frequenza}

Consideriamo la funzione di trasferimento:
\begin{equation}
    G(s) = \frac{2.407 \cdot 10^{-5}}{s^2 + 1.968s + 8.509 \cdot 10^{-3}}
\end{equation}

Per passare al dominio della frequenza, sostituiamo $s$ con $j\omega$:
\begin{equation}
    G(j\omega) = \frac{2.407 \cdot 10^{-5}}{(j\omega)^2 + 1.968(j\omega) + 8.509 \cdot 10^{-3}}
\end{equation}\\
Notiamo che ci troviamo nel caso di un sistema del secondo ordine (due poli a parte reale negativa). Di conseguenza il sistema si comporta come un filtro passa-basso reale. \\
Il guadagno statico si mantiene costante fino a quando la frequenza raggiunge il modulo del polo più vicino all'origine, ovvero $\omega = 0.0043$ rad/s. A questa frequenza, il polo $p_2$ introduce una pendenza di $-20$ dB/decade nel diagramma delle ampiezze. 
Dopo che anche il secondo polo si manifesta ad $\omega = 1.9641$ rad/s avremo una pendenza di $-40$ db/decade.\\
Per quanto riguarda il diagramma della fase, vediamo che presenta due curvature principali, in corrispondenza dei due poli, per poi assestarsi a frequenze alte su $-180$°.
\section{Mappatura specifiche del regolatore}
\label{sec:specifications}

Le specifiche da soddisfare sono le seguenti:\\ \\
\textbf{Specifica 1:} Errore a regime $|e_\infty| \le e^* = 0.001$ in risposta ad un gradino
$w(t) = W \cdot 1(t)$ e $d(t) = D \cdot 1(t)$ con ampiezze $W \le 50$ e $D \le 2$ \\

Sappiamo che
\begin{align*}
	e_\infty &= \lim_{t \rightarrow \infty} e(t) = \lim_{s \rightarrow 0} sE(s) = \lim_{s \rightarrow 0} sS(s)(\dfrac{W}{s} + \dfrac{D}{s}) = (W + D) \lim_{s \rightarrow 0} S(s)
\end{align*}

Vista l'assenza di poli in $s=0$, il limite di $S(s)$ per $s \rightarrow 0$ vale $\frac{1}{1 + \mu}$, e quindi
\begin{align*}
	e_\infty = \frac{W+D}{1+\mu} \approx \frac{W+D}{\mu} \le e^*
\end{align*} 

e isolando $\mu$
\begin{align*}
	\mu = L(0) \ge \frac{D^* + W^*}{e^*} = 5.2 \cdot 10^4
\end{align*}

Abbiamo quindi ottenuto la prima condizione:
\begin{align*}
	\setlength{\fboxsep}{10pt}
	\Aboxed{\mu \ge 5.2 \cdot 10^4}
\end{align*}
\\
\textbf{Specifica 2:} Margine di fase $M_f \ge 40$° \\ \\
\textbf{Specifica 3:} Sovraelongazione percentuale massima del $18\%$, quindi $S\% \le 18\% $ \\

Sappiamo che $S$ \% $\le S^{\star}$ equivale a richiedere $\xi \ge\xi^{\star}$, con $S^{\star} = e^{\frac{-\pi\xi^{\star}}{\sqrt{1-(\xi^{\star})^2}}}$. Procedendo con i calcoli si ottiene:
\begin{align*}
    \xi^{\star} =0.479133 \approx 0.48
\end{align*} 
Essendo $M_f \ge 100\xi^{\star}$, allora si ottiene la condizione  
\begin{align*}
	\setlength{\fboxsep}{10pt}
	\Aboxed{M_f \ge 48\%}
\end{align*}
La quale, dovendo scegliere la più restrittiva tra la specifica 2 e 3, risulta la condizione da imporre sul margine di fase. 
\\ \\
\textbf{Specifica 4:} Il tempo di assestamento alla $\epsilon\% = 1\%$ deve essere inferiore al valore fissato: $T{a,\epsilon}=0.3s$.\\

Ricordando che $T_{a,1} = 4.6T$ e che $T=\frac{1}{\xi\omega_c}$, allora si ottiene che
\begin{align*}
    \omega_c\ge\frac{460}{T^{\star}M_f^{\star}} \approx 32\text{ rad/s}
\end{align*}
Possiamo concludere che 
\begin{align*}
	\setlength{\fboxsep}{10pt}
	\Aboxed{\omega_{c,min}= 32 {\text{ rad/s}}}
\end{align*}
\\
\textbf{Specifica 5:} Il disturbo sull'uscita $d(t)$, con banda limitata nel range delle pulsazioni $[0, 0.1]$ deve essere abbattuto di almeno 60 dB; ciò equivale a: 
\begin{align*}
    |L(j\omega)|_{\text{dB}} \ge60 {\text{ dB \quad con } \omega\in[\omega_{d,min},\omega_{d,max}]=[0,0.1]}
\end{align*}\\ \\
\textbf{Specifica 6:} Il rumore di misura $n(t)$, con banda limitata nel range delle pulsazioni $[10^3, 10^6]$ deve essere abbattuto di almeno 40 dB; ciò equivale a:
\begin{align*}
    |L(j\omega)|_{\text{dB}} \le-40 {\text{dB\quad con }\omega\in[\omega_{n,min},\omega_{n,max}]=[10^3,10^6]}
\end{align*}
\\ \\
%
Pertanto, in Figura \ref{fig:mappatura} mostriamo il diagramma di Bode della funzione di trasferimento $G(s)$ con le zone proibite emerse dalla mappatura delle specifiche.\\

\begin{figure}[h!]
    \centering
    \includegraphics [width=0.9\textwidth]{../figs/mappatura.pdf} 
    \caption{Mappatura delle richieste}
    \label{fig:mappatura}
\end{figure}
Si può notare dal diagramma delle ampiezze che il guadagno statico è insufficiente. E' compito del regolatore alzarlo per soddisfare la specifica 1.
\section{Sintesi del regolatore statico}
\label{sec:static_regulator}

In questa sezione progettiamo il regolatore statico $R_s(s)$ partendo dalle analisi fatte in sezione~\ref{sec:specifications}.\\

La specifica 1 sull'errore a regime impone di avere un guadagno statico pari o superiore a $\mu^{\star}_e=5.2\cdot10^4$. 
Di conseguenza, il guadagno del regolatore statico deve essere tale da imporre un guadagno complessivo almeno pari a $\mu^{\star}_e$. 
Ne consegue che il guadagno di $R_s(s)$ deve essere pari a:
\begin{align*}
    \mu_e= \frac{\mu^{\star}_e}{|G(0)|} = \frac{5.2\cdot10^4}{2.8\cdot10^{-3}}= 1.8382\cdot10^7
\end{align*}\\

La specifica 5 sul disturbo in uscita impone di avere 
\begin{align*}
    \min_{[\omega_{d,min},\omega_{d,max}]}|L(j\omega)|_{\text{dB}}=|L(\omega_{d,max})|_{\text{dB}}=|G(\omega_{d,max})|_{\text{dB}} \ge60{\text{dB}}=\mu^{\star}_d
\end{align*}
Passando dai decibel a una forma lineare otteniamo:
\begin{align*}
    \mu^{\star}_d=10^{\frac{60}{20}}=10^3
\end{align*}
Di conseguenza, il guadagno del regolatore statico deve essere tale da imporre un guadagno complessivo almeno pari a $\mu^{\star}_d$, che è pari a:
\begin{align*}
    \mu_d= \frac{\mu^{\star}_d}{|G(\omega_{d,max})|} = \frac{10^3}{1.2229\cdot10^{-4}}= 8.1775\cdot10^6
\end{align*}\\

In definitiva, dobbiamo imporre che il guadagno statico di $R_s(s)$ sia:
\begin{align*}
    \mu= \max\{\mu_e, \mu_d\} = \mu_e=1.8382\cdot10^7
\end{align*}\\
Non dovendo soddisfare altri requisiti, termina qui il progetto del regolatore statico, la cui espressione è semplicemente:
\begin{align*}
    \setlength{\fboxsep}{10pt}
    \Aboxed{R_s(s)=\mu}
\end{align*}\\

Dunque, definiamo la funzione estesa $G_e(s) = R_s(s)G(s)$ e, in Figura \ref{fig:G_estesa}, mostriamo il suo diagramma di Bode per verificare se e quali zone proibite vengono attraversate.\\
\begin{figure}[h!]
    \centering
    \includegraphics [width=0.9\textwidth]{../figs/G_estesa.pdf} 
    \caption{Funzione di trasferimento estesa}
    \label{fig:G_estesa}
\end{figure}
\\Dalla Figura \ref{fig:G_estesa}, emerge che nel diagramma di ampiezza la funzione estesa interseca la zona proibita corrispondente alla specifica 4 sulla frequenza critica minima, ossia la frequenza di attraversamento a $0$ dB è inferiore a $\omega_{c,min}$.\\
Inoltre possiamo notare dal diagramma di fase che nell'intervallo di pulsazioni ammissibili per la pulsazione di attraversamento $\omega_c$ non esistono pulsazioni in cui la fase rispetta il vincolo sul margine di fase, relativo alle specifiche 2 e 3.

\section{Sintesi del regolatore dinamico}

In questa sezione, progettiamo il regolatore dinamico $R_d(s)$.\\
Dalle analisi fatte in Sezione~\ref{sec:static_regulator}, notiamo di essere nello Scenario di tipo B. Dunque, progettiamo $R_d(s)$ riccorrendo a una rete anticipatrice, ossia una rete nella forma:
\begin{align*}
    R_d(s) =\frac{1+\tau s}{1+\alpha\tau s}\qquad0<\alpha<1
\end{align*}\\
In particolare, applichiamo le formule di inversione per ottenere i parametri $\tau$ e $\alpha\tau$, imponendo la frequenza di attraversamento come $\omega_c^{\star}=70\space$ rad/s. Inoltre, inseriamo un margine di sicurezza per la fase $\varepsilon=8$. Chiamiamo $M^{\star}>1$ l'amplificazione della rete anticipatrice a $\omega_c^{\star}$ e $0<\varphi^{\star}<\frac{\pi}{2}$ lo sfasasmento a tale pulsazione. Allora bisogna che sia:
\begin{align*}
    M^{\star} =\frac{1}{|G_e(j\omega_c^{\star})|}\qquad\varphi^{\star}=M_f^{\star}-[180\degree+\arg\{G_e(j\omega_c^{\star})\}]+\varepsilon
\end{align*}\\
Quindi, tramite le formule di inversione:
\begin{align*}
    \tau =\frac{M^{\star}-\cos\varphi^{\star}}{\omega_c^{\star}\sin\varphi^{\star}}=0.1846 \qquad \alpha\tau =\frac{\cos\varphi^{\star}-\frac{1}{M^{\star}}}{\omega_c^{\star}\sin\varphi^{\star}}=8.7\cdot10^{-3}
\end{align*}\\
In Figura \ref{fig:L_anello} mostriamo il diagramma di Bode della funzione d'anello $L(s) = R_d(s) G_e(s)$.

\begin{figure}[h!]
    \centering
    \includegraphics [width=0.9\textwidth]{../figs/L_anello.pdf} 
    \caption{Funzione d'anello $L(s)$}
    \label{fig:L_anello}
\end{figure}
Possiamo notare che tutte le specifiche sono rispettate e di conseguenza una rete anticipatrice è sufficiente per il progetto del regolatore dinamico.

\section*{Luogo delle radici}

Avendo definito $L(s)$, possiamo fare alcune considerazioni sul luogo delle radici del sistema in anello aperto.
Partiamo dalla sua espressione\\
\begin{align}
	L(s) = R_d(s)G_e(s) = 52000 \dfrac{1+ 0.1846s}{(1+0.00868s)(1+0.5091s)(1+230.8s)}
\end{align}\\
Osserviamo che ha uno zero ($m=1$) e tre poli ($n=3$) rispettivamente in
\begin{align*}
	z_1 = -5.417 \qquad p_1 = -115.24 \qquad p_2 = -1.964 \qquad p_3 = -0.0043
\end{align*}
Usando le regole di tracciamento, possiamo già trarre alcune
conclusioni immediate sul luogo delle radici del sistema, e cioè:
\begin{itemize}
	\item Il luogo ha $3$ rami. Di questi, uno soltanto termina in uno zero, mentre gli altri due terminano all'infinito.
	\item I soli punti dell'asse reale che appartengono al luogo sono quelli che lasciano alla propria destra un numero pari di singolarità, e cioè i punti compresi tra $p_1$ e $z_1$, e quelli tra $p_2$ e $p_3$ 
	\item L'asintoto lungo il quale uno dei 3 rami tende all'infinito, interseca l'asse reale nel punto con ascissa pari a $$ x_a = \frac{1}{2}(-z_1+\sum_{i=1}^3 p_i) = -55.898 $$ ovvero tra $p_1$ e $z_1$.
	\item In particolare, l'asintoto interseca l'asse reale formando un angolo pari a $$ \theta_{a,0} = \frac{\pi}{n-m} = \frac{\pi}{2} $$ quindi l'asse reale e l'asintoto sono perpendicolari.
\end{itemize}

A questo punto possiamo procedere nel tracciamento del luogo delle radici, seguendo le considerazioni fatte sopra.\\
Il risultato finale è illustrato in Figura \ref{fig:luogo_radici}.

\begin{figure}[h!]
    \centering
    \includegraphics [width=0.9\textwidth]{../figs/luogo_radici.pdf} 
    \caption{Luogo delle radici di $L(s)$}
    \label{fig:luogo_radici}
\end{figure}

\section{Test sul sistema linearizzato}

In questa sezione testiamo l'efficacia del controllore progettato sul sistema linearizzato con 
\begin{align}
w(t) = 50 \cdot 1(t) \qquad d(t) = 0.8 \sum_{k=1}^4 \text{sin}(0.02kt) \qquad n(t) = 0.5 \sum_{k=1}^4 \text{sin}(10^5kt). 
\end{align}

Definiamo la funzione di sensitività $S(s)$ e di sensitività complementare $F(s)$ come
\begin{align}
	S(s) = \dfrac{1}{1+L(s)} \qquad F(s) = \dfrac{L(s)}{1+L(s)} 
\end{align}

Inoltre grazie alla sovrapposizione degli effetti possiamo studiare la risposta ai segnali separatamente. Si ha infatti che
\begin{align*}
	Y(s) = Y_w(s) + Y_d(s) + Y_n(s)
\end{align*}
Iniziamo dal primo caso, cioè $Y_w(s)$. Trasformando secondo Laplace $w(t) = 50 \cdot 1(t)$ otteniamo $W(s) = \dfrac{50}{s}$.
Di conseguenza 
\begin{align*}
	Y_w(s) = F(s) \cdot W(s) = \dfrac{50 \cdot F(s)}{s}
\end{align*}
In Figura \ref{fig:bode_diagram_L} possiamo notare che vengono rispettate le specifiche 3 e 4 riguardo alla sovraelongazione percentuale e al tempo di assestamento.

\begin{figure}[h!]
    \centering
    \includegraphics [width=0.9\textwidth]{../figs/risposta_gradino.pdf} 
    \caption{Risposta del sistema al gradino $w(t)$}
    \label{fig:bode_diagram_L}
\end{figure}

Proseguiamo considerando $Y_d(s)$. La trasformata di Laplace del disturbo $d(t)$ vale
\begin{align*}
	D(s) = 0.8 \sum_{k=1}^4 \dfrac{\omega_d}{s^2 + \omega_d^2}
\end{align*}

con $\omega_d = 0.02k$. Allora otteniamo
\begin{align*}
	Y_d(s) = S(s) \cdot D(s) = 0.8 \cdot S(s) \sum_{k=1}^4 \dfrac{\omega_d}{s^2 + \omega_d^2}
\end{align*}

\begin{figure}[h!]
    \centering
	\includegraphics [width=0.9\textwidth]{../figs/risposta_d.pdf} 
	\caption{Risposta a $d(t)$}
    \label{fig:risposta_d}
\end{figure}

Analogamente, trasformiamo secondo Laplace il disturbo $n(t)$:
\begin{align*}
	N(s) =  0.5 \sum_{k=1}^4 \dfrac{\omega_n}{s^2 + \omega_n^2}
\end{align*}
con $\omega_n=10^5k$. 
Di conseguenza si ottiene
\begin{align*}
	Y_n(s) =  -F(s)\cdot N(s) = - 0.5\cdot F(s) \sum_{k=1}^4 \dfrac{\omega_n}{s^2 + \omega_n^2}
\end{align*}

\begin{figure}[h!]
    \centering
	\includegraphics [width=0.9\textwidth]{../figs/risposta_n.pdf} 
	\caption{Risposta a $n(t)$}
    \label{fig:risposta_n}
\end{figure}

In Figura \ref{fig:risposta_n} è illustrata la risposta del sistema al disturbo $n(t)$.
Come possiamo notare, in entrambi i casi l’uscita viene quasi totalmente abbattuta. Questo accade perché nella progettazione del nostro regolatore abbiamo sfruttato una caratteristica importante dei due disturbi $d(t)$ e $n(t)$: entrambi hanno le proprie bande limitate in opportuni range. $d(t)$
infatti ha bande a bassa frequenza, mentre $n(t)$ lavora con bande ad alta frequenza.
Il regolatore è stato costruito in modo da avere $S(j\omega)$ $\ll$ $1$ a basse frequenze e $F(j\omega) \ll 1$ ad alte frequenze, provando in questo modo ad abbattere entrambi i disturbi. I risultati ottenuti sono quindi in accordo con le specifiche del problema.
\newpage
\section{Test sul sistema non lineare}

In questa sezione, testiamo l'efficacia del controllore progettato sul modello non lineare su Simulink.

In Figura \ref{fig:sistema_non_lineare}, mostriamo lo schema a blocchi del sistema in anello chiuso. Per ottenere il sistema non linearizzato sono stati inseriti i seguenti componenti:
\begin{itemize}
    \item Un blocco che genera una funzione a gradino ad ampiezza 10
    \item Un blocco che contiene la funzione del regolatore R
    \item Un blocco che contiene la funzione $f(x,u)$
    \item Un integratore che permette di ottenere $x$ a partire dall'uscita $\dot{x}$ del blocco di $f(x,u)$. Esso è posto in retroazione con il blocco Matlab per utilizzarlo come suo ingresso. Inoltre questo blocco permette di inserire i valori iniziali dello stato.
    \item Il gain moltiplica il valore di uscita dell'integratore per la matrice $C$
    \item Lo scope sull'uscita permette di visualizzarla
\end{itemize}

\begin{figure}[h!]
    \centering
	\includegraphics [width=0.9\textwidth]{../figs/simulatore_non_lineare.pdf} 
	\caption{Sistema non lineare in Simulink}
	\label{fig:sistema_non_lineare}
\end{figure}

In Figura \ref{fig:risposta_gradino_simulink} è riportata la risposta del sistema non linearizzato a fronte di un ingresso a gradino $w(t)$ con ampiezza 10. Possiamo notare facilmente che, a differenza del sistema linearizzato, non rispetta le specifiche del problema riguardanti il tempo di assestamento, evidenziando le complicazioni che nascono con un modello non linearizzato.

\begin{figure}[h!]
    \centering
	\includegraphics [width=0.9\textwidth]{../figs/risposta_non_linearizzato.pdf} 
	\caption{Risposta del sistema non lineare al gradino $w(t) = 10\cdot 1(t)$}
	\label{fig:risposta_gradino_simulink}
\end{figure}

Abbiamo esplorato il range di condizioni iniziali tali per cui l’uscita del sistema in anello chiuso converge a $h(x_e, u_e)$ e le abbiamo rappresentate tramite due heatmap in Figura \ref{fig:heatmap}: una per il tempo di assestamento e una per la sovraelongazione. 
Possiamo notare come entrambe dipendano principalmente da $x_{2,e}$, ossia dalla temperatura dell'aria in uscita dal riscaldatore, mentre non dipendono molto da $x_{1,e}$, ossia dalla temperatura del riscaldatore stesso.
In particolare, abbiamo stabilità per valori sufficientemente grandi di $x_{2,e}$. All'aumentare di $x_{2,e}$, osserviamo un leggero aumento del tempo di assestamento e una diminuzione della sovraelongazione percentuale.\\

\begin{figure}[h!]
    \centering
	\includegraphics [width=0.9\textwidth]{../figs/Esplorazione_range_equilibri.pdf} 
	\caption{Heatmap dei tempi di assestamento al variare dello stato iniziale del sistema}
	\label{fig:heatmap}
\end{figure}

Infine abbiamo esplorato il range di ampiezze dei riferimenti a gradino tali per cui il regolaratore rimanesse efficace sul sistema non lineare.
Innanzitutto, abbiamo testato ampiezze positive. Da tali test è emerso che il sistema si stabilizza indipendentemente dall'ampiezza del gradino, ma sovraelongazione e tempo di assestamento sono direttamente proporzionali all'ampiezza del gradino.
Successivamente abbiamo provato a utilizzare ampiezze negative per il riferimento a gradino, e i test hanno portato a concludere che circa fino a un'ampiezza pari a -19 il sistema diverge e di conseguenza il regolatore non è più efficace. 
Per valori inferiori, invece, il sistema torna a convergere, ma con tempi di assestamento molto dilatati e verso un valore pari circa a 16.
La Figura \ref{fig:ampiezze} riassume queste considerazioni.

\begin{figure}[h!]
    \centering
	\includegraphics [width=1.0\textwidth]{../figs/confronto_ampiezze.pdf} 
	\caption{Risposte del sistema ad impulsi con diversa ampiezza.}
	\label{fig:ampiezze}
\end{figure}

% inserire snapshot
\clearpage
\section{Punto opzionale: animazione}

Abbiamo realizzato in MATLAB un'animazione del sistema, graficando l'evoluzione lungo un periodo di 10
secondi di $T_{out}$ rispetto al riferimento a gradino con ampiezza 10 e l'evoluzione dell'errore di regolazione. In tale animazione mostriamo anche la potenza $P_E$ in ingresso al sistema.
In Figura \ref{fig:screenshot} abbiamo riportato uno screenshot dell'animazione. 

\begin{figure}[h!]
    \centering
	\includegraphics [width=0.9\textwidth]{../figs/animazione.pdf} 
	\caption{Interfaccia grafica realizzata in Matlab.}
	\label{fig:screenshot}
\end{figure}

\section{Conclusioni}

Andremo in questo breve capitolo conclusivo a sintetizzare quanto fatto.

Iniziando con il lavoro svolto nel primo punto, attraverso un processo di linearizzazione nell’intorno del punto di equilibrio siamo riusciti ad analizzare un sistema non lineare. Trasformando quindi le equazioni che caratterizzano il sistema non lineare ci è stato possibile studiarne il comportamento.
Tramite il calcolo della funzione di trasferimento e l’analisi in frequenza, abbiamo evidenziato le proprietà del nostro sistema.

A questo punto, partendo dalla mappatura delle specifiche fornite abbiamo progettato un regolatore formato dal prodotto di una parte statica e una parte dinamica. La prima si è occupata di imporre il guadagno richiesto, la seconda delle restanti specifiche. In particolare, nel regolatore dinamico ci è stata sufficiente una rete anticipatrice, senza l'introduzione di un polo ad alta frequenza.

Conoscendo le caratteristiche delle varie funzioni di sensività, abbiamo testato il sistema ottenuto con 3 ingressi differenti: un ingresso a gradino e disturbi d’uscita e di misura sinusoidali. Con le rappresentazioni grafiche delle risposte abbiamo appurato come anche in questo caso le specifiche siano state rispetatte e, in particolare, come i due rumori siano stati quasi totalmente abbattuti garantendo ulteriormente la robustezza del regolatore progettato.

Nel quinto punto del progetto il sistema non linearizzato è stato ricostruito su SimuLink. Questo test ha evidenziato l’approssimazione della linearizzazione svolta, in quanto si nota il tempo di assestamento in questo caso molto maggiore.
Inoltre abbiamo studiato in maniera più completa le performance del sistema esplorando il range di condizioni iniziali dello stato e e di ampiezza di riferimenti a gradino per cui il sistema risultava efficace. 

Infine il punto opzionale ci ha permesso di avere un riscontro animato del comportamento del nostro scaldatore. 

\end{document}